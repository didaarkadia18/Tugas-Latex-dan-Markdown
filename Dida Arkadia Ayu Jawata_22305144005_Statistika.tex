\documentclass[a4paper,10pt]{article}
\usepackage{eumat}

\begin{document}
\begin{eulernotebook}
\begin{eulercomment}
NAMA  : Dida Arkadia Ayu Jawata\\
NIM   : 22305144005\\
KELAS : Matematika E 2023\\
\begin{eulercomment}
\eulerheading{Menggambar Grafik  Statistika}
\begin{eulercomment}
\end{eulercomment}
\eulersubheading{Diagram Kotak}
\begin{eulercomment}
Diagram kotak atau box plot merupakan ringkasan distribusi sampel yang
disajikan secara grafis yang bisa menggambarkan bentuk distribusi data
(skewness), ukuran tendensi sentral dan ukuran penyebaran (keragaman)
data pengamatan. Diagram kotak sering digunakan ketika  jumlah
distribusi data perlu dibandingkan. Diagram kotak menyajikan informasi
tentang nilai--nilai inti dalam distribusi data termasuk juga
pencilan. Pencilan adalah titik data yang terpaut jauh dari titik data
lainnya.

Contoh:\\
Diketahui data berat badan mahasiswa di Universitas A sebagai berikut.
\end{eulercomment}
\begin{eulerprompt}
>A=[55,50,33,42,44,37,63,74,56,34,51,43,45,39,64,77,60,35,53,43,48,41,65,87,61,36,54,44,49,41,66,89]
\end{eulerprompt}
\begin{euleroutput}
  [55,  50,  33,  42,  44,  37,  63,  74,  56,  34,  51,  43,  45,  39,
  64,  77,  60,  35,  53,  43,  48,  41,  65,  87,  61,  36,  54,  44,
  49,  41,  66,  89]
\end{euleroutput}
\begin{eulercomment}
Buatlah diagram kotak (box plot) kemudian tuliskan interpretasinya.
\end{eulercomment}
\begin{eulerprompt}
>boxplot(A):
\end{eulerprompt}
\eulerimg{29}{images/Dida Arkadia Ayu Jawata_22305144005-001.png}
\begin{eulercomment}
Dari gambar box plot berat  badan mahasiswa Universitas  A, sepintas
kita bisa menentukan beberapa ukuran statistik, meskipun tidak persis
sekali. Nilai statistik pada badan boxplot berkisar pada: Nilai
Minimum = 33 , Q1 = 41.5 , Median (Q2) = 49.5 , Q3 = 62 , Nilai
Maksimum  = 89 . Sebaran data tidak simetris, melainkan menjulur ke
arah kanan (postively skewness). Karena nilai jarak Q1 dengan Q2 lebih
pendek dari jarak Q2 dengan Q3, maka data lebih terpusat di kiri. Akan
tetapi data tersebut tergolong cenderung mesokurtik karena jarak IQR
dengan panjang hampir sama, dengan data berpusat di angka 49.5


Adapun contoh perbandingan 10 simulasi 500 nilai terdistribusi normal
menggunakan box plot dan terdapat pencilan sebagai berikut.

\end{eulercomment}
\begin{eulerprompt}
> p=normal(10,500); boxplot(p):
\end{eulerprompt}
\eulerimg{29}{images/Dida Arkadia Ayu Jawata_22305144005-002.png}
\begin{eulercomment}
pada diagram diatas, adalah membuat boxplot distribusi normal dengan
rata-rata 10 dan standar deviasi 500. Boxplot adalah representasi
grafis dari lokalitas, penyebaran, dan kecondongan sekelompok data
numerik melalui kuartil mereka\\
2

\end{eulercomment}
\eulersubheading{Diagram Batang}
\begin{eulercomment}
Diagram batang adalah representasi visual dari data yang menggunakan
balok atau kolom vertikal untuk mewakili kategori, nilai atau variabel
tertentu. Setiap kolom yang ada pada diagram  batang memiliki
frekuensi atau jumlah dalam kategori tersebut.

Contoh:

Kita akan membuat diagram batang secara random.
\end{eulercomment}
\begin{eulerprompt}
>columnsplot(cumsum(random(6)),style="/",color=red):
\end{eulerprompt}
\eulerimg{29}{images/Dida Arkadia Ayu Jawata_22305144005-003.png}
\begin{eulerprompt}
>columnsplot(cumsum(random(15)),style="-",color=black):
\end{eulerprompt}
\eulerimg{29}{images/Dida Arkadia Ayu Jawata_22305144005-004.png}
\begin{eulerprompt}
>columnsplot(cumsum(random(3)),style="|",color=orange):
\end{eulerprompt}
\eulerimg{29}{images/Dida Arkadia Ayu Jawata_22305144005-005.png}
\begin{eulercomment}
Selanjutnya kita akan mencoba  membuat diagram batang penjualan yang
menggunakan variabel.
\end{eulercomment}
\begin{eulerprompt}
>months=["Januari","Februari","Maret","April","Mei"];
>values=[20,50,40,70,30];
>columnsplot(values,lab=months,color=yellow);
>title("Data Penjualan Beras Toko Kuning pada tahun 2023"):
\end{eulerprompt}
\eulerimg{29}{images/Dida Arkadia Ayu Jawata_22305144005-006.png}
\begin{eulercomment}
Perintah "columnsplot(values,lab=months,color=yellow);" merupakan
sintaks untuk membuat diagram batang dengan menggunakan nilai dari
variabel "values", label bulan dari variabel "months", dan warna
kuning

Dari diagram batang tersebut kita bisa mengetahui data penjualan toko
kuning selama lima bulan pada tahun 2023  yaitu, pada bulan Januari,
Februari, Maret , April, Mei. Januari terjual 20 ton beras, Februari
terjual 50 ton beras, Maret terjual 40 ton beras, April terjual 70 ton
beras, dan Mei terjual 30 ton beras.

\end{eulercomment}
\eulersubheading{Diagram Lingkaran}
\begin{eulercomment}
Diagram lingkaran merupakan penyajian statistik data tunggal dalam\\
bentuk lingkaran yang dibagi menjadi beberapa juring atau sektor yang\\
menggambarkan banyak frekuensi untuk setiap data.Diagram lingkaran\\
tidak menampilkan informasi frekuensi dari masing-masing data secara\\
detail.
\end{eulercomment}
\begin{eulerprompt}
>CP:=[rgb(0.5,0.5,0.5),red,yellow,green,rgb(0.9,0,0)]
\end{eulerprompt}
\begin{euleroutput}
  [5.87532e+07,  2,  15,  3,  6.54049e+07]
\end{euleroutput}
\begin{eulerprompt}
>i=[1,2,3,4,5]; piechart(values[i],color=CP[i],lab=months[i]):
\end{eulerprompt}
\eulerimg{29}{images/Dida Arkadia Ayu Jawata_22305144005-007.png}
\begin{eulercomment}
RGB adalah singkatan dari Red, Green, and Blue, dan setiap parameter
mendefinisikan intensitas warna dengan nilai antara 0 dan 1. Warna
pertama dalam daftar adalah warna abu-abu dengan jumlah merah, hijau,
dan biru yang sama. Warna kedua merah, ketiga kuning, dan keempat
hijau. Warna terakhir adalah warna merah dengan lebih banyak merah
daripada hijau atau biru.

\end{eulercomment}
\eulersubheading{Diagram Bintang}
\begin{eulercomment}
Diagram bintang, terkadang disebut diagram radar atau diagram web,
adalah metode perangkat grafis yang digunakan untuk menampilkan data
multivariat. Multivariat dalam pengertian ini mengacu pada memiliki
banyak karakteristik untuk diamati. Variabelnya juga harus berupa
nilai yang berkisar.\\
Diagram bintang terdiri dari rangkaian jari-jari bersudut sama, yang
disebut jari-jari, dengan masing-masing jari mewakili salah satu
variabel. Panjang jari-jari data sebanding dengan besaran variabel
pada titik data relatif terhadap besaran maksimum variabel di
seluruh titik data.
\end{eulercomment}
\begin{eulerprompt}
>starplot(normal(1,15)+16,lab=1:15,>rays):
\end{eulerprompt}
\eulerimg{29}{images/Dida Arkadia Ayu Jawata_22305144005-008.png}
\begin{eulerprompt}
>starplot(values,lab=months,>rays):
\end{eulerprompt}
\eulerimg{29}{images/Dida Arkadia Ayu Jawata_22305144005-009.png}
\begin{eulercomment}
Syntax starplot(values,lab=months,rays) adalah perintah untuk membuat
grafik bintang (star plot) dengan menggunakan nilai-nilai yang
diberikan dalam vektor values, label sumbu yang diberikan dalam vektor
months, dan jumlah rays yang menentukan jumlah garis radial yang
digunakan dalam grafik


\end{eulercomment}
\eulersubheading{Diagram Impuls}
\begin{eulercomment}
Impuls (impulse) adalah perubahan momentum. Contohnya adalah sebuah
bola bermassa yang tengah ditendang, bola menggelinding yang
dihentikan, bola jatuh yang memantul, mobil yang menabrak tembok,
telur jatuh yang pecah.\\
Berikut adalah plot impuls dari data acak 1 sampai 20, terdistribusi
secara merata di [0,1].
\end{eulercomment}
\begin{eulerprompt}
>plot2d(makeimpulse(1:20,random(1,20)),>bar):
\end{eulerprompt}
\eulerimg{29}{images/Dida Arkadia Ayu Jawata_22305144005-010.png}
\begin{eulercomment}
Tetapi untuk data yang terdistribusi secara eksponensial, kita mungkin
memerlukan plot logaritmik.
\end{eulercomment}
\begin{eulerprompt}
> logimpulseplot(1:20,-log(random(1,20))*10):
\end{eulerprompt}
\eulerimg{29}{images/Dida Arkadia Ayu Jawata_22305144005-011.png}
\begin{eulercomment}
Jadi gambar grafiknya terlihat naik turun (mengalami perubahan).
\end{eulercomment}
\eulersubheading{Histogram}
\begin{eulercomment}
Histogram adalah representasi grafis (diagram) yang mengatur dan
menampilkan frekuensi data sampel pada rentang tertentu. Frekuensi
data yang ada pada masing-masing kelas direpresentasikan dengan bentuk
grafik diagram batang atau kolom.
\end{eulercomment}
\begin{eulerprompt}
>aspect(1); plot2d(random(100),>histogram):
\end{eulerprompt}
\eulerimg{29}{images/Dida Arkadia Ayu Jawata_22305144005-012.png}
\begin{eulerprompt}
>r=150:5:185; v=[22,71,136,150,139,71,32];
>plot2d(r,v,a=150,b=185,c=0,d=150,bar=1,style="/"):
\end{eulerprompt}
\eulerimg{29}{images/Dida Arkadia Ayu Jawata_22305144005-013.png}
\begin{eulercomment}
Pola "r=150:5:185" berarti bahwa nilai r dimulai dari 150, kemudian
bertambah 5 setiap kali, dan berakhir saat mencapai atau melebihi 185.
Dengan pola ini, kita dapat menentukan nilai-nilai r yang sesuai.

Dari data yang diperoleh dapat diketahui bahwa dari rentang kelas
150-155 memiliki frekuensi 22, rentang kelas 155-160 memiliki
frekuensi 71,  dan seterusnya.

\end{eulercomment}
\eulersubheading{Kurva Fungsi Kerapatan Probabilitas}
\begin{eulercomment}
Secara teoritis kurva probabilitas populasi diwakili oleh poligon
frekuensi relatif yang dimuluskan (variabel acak  kontiniu
diperlakukan seperti variabel acak diskrit yang rapat).Karena itu
fungsi dari variabel acak kontiniu merupakan fungsi kepadatan
probabilitas (probability density function – pdf). Pdf menggambarkan
besarnya probabilitas per unit interval nilai variabel acaknya.
\end{eulercomment}
\begin{eulerprompt}
>plot2d("qnormal(x,0,1)",-5,5);  ...
>plot2d("qnormal(x,0,1)",a=1,b=4,>add,>filled):
\end{eulerprompt}
\eulerimg{29}{images/Dida Arkadia Ayu Jawata_22305144005-014.png}
\begin{eulercomment}
Perintah "plot2d("qnormal(x,0,1)",-5,5)" digunakan untuk membuat plot
dari distribusi normal dengan mean 0 dan standard deviation 1 di
rentang -5 hingga 5

Probabilitas variabel acak x yang terletak antara 1 dan 4 memenuhi\\
P(1\textless{}X\textless{}4)= luas daerah hijau

\end{eulercomment}
\eulersubheading{Kurva Fungsi Distribusi Kumulatif}
\begin{eulercomment}
Cumulative Distribution Function (CDF) atau fungsi distribusi
kumulatif adalah fungsi matematika yang digunakan untuk menghitung
probabilitas variabel acak diskrit atau kontinu. CDF memberikan
probabilitas bahwa variabel acak akan menghasilkan nilai kurang dari
atau sama dengan nilai tertentu. Dalam hal ini, CDF dapat digunakan
untuk menghitung probabilitas kumulatif dari variabel acak.

Berikut merupakan contoh kurva fungsi distribusi kumulatif kontinu:
\end{eulercomment}
\begin{eulerprompt}
>splot2d("normaldis",-3,5):
\end{eulerprompt}
\begin{euleroutput}
  Function splot2d not found.
  Try list ... to find functions!
  Error in:
  splot2d("normaldis",-3,5): ...
                           ^
\end{euleroutput}
\begin{eulercomment}
Dapat kita lihat dalam kurva fungsi distribusi kumulatif kontinu
terdiri atas tiga bagian yaitu:\\
1. Bernilai 0 untuk x di  bawah minimal dari daerah rentang.\\
2. Merupakan fungsi monoton naik pada daerah rentang.\\
3. Mempunyai nilai konstan 1 di atas batas maksimum daerah rentangnya.

Adapun contoh kurva fungsi distribusi kumulatif diskrit sebagai
berikut.
\end{eulercomment}
\begin{eulerprompt}
>x=normal(1,6);
\end{eulerprompt}
\begin{eulercomment}
Baris kode tersebut akan menghasilkan suatu nilai acak dari distribusi
normal dengan mean 1 dan deviasi standar 6, dan nilai tersebut
disimpan dalam variabel x. Variabel x kemudian dapat digunakan dalam
perhitungan atau analisis selanjutnya

Fungsi empdist(x,vs) membutuhkan array nilai yang diurutkan. Jadi kita
harus mengurutkan x sebelum kita dapat menggunakannya.
\end{eulercomment}
\begin{eulerprompt}
>xs=sort(x);
>plot2d("empdist",-3,5;xs):
\end{eulerprompt}
\eulerimg{29}{images/Dida Arkadia Ayu Jawata_22305144005-015.png}
\begin{eulercomment}
Grafik fungsi distribusi kumulatif peubah acak diskrit merupakan
fungsi tangga naik dengan nilai terendah 0 dan nilai tertinggi 1.
\end{eulercomment}
\end{eulernotebook}
\end{document}
