\documentclass[a4paper,10pt]{article}
\usepackage{eumat}

\begin{document}
\begin{eulernotebook}
\begin{eulercomment}
NAMA : Dida Arkadia Ayu Jawata\\
NIM : 22305144005\\
KELAS  : Matematika E
\end{eulercomment}
\begin{eulercomment}

\begin{eulercomment}
\eulerheading{PLOT 2D}
\eulerheading{Menggambar Grafik Fungsi Simbolik}
\begin{eulercomment}
Fungsi Plot yang paling penting untuk plot planar adalah plot2d().
Fungsi ini diimplementasikan dalam bahasa Euler dalam file "plot.e",
yang dimuat diawal program.

plot2d() menerima ekspresi, fungsi, dan data.

Rentang plot diatur dengan parameter yang ditetapkan ssbagai berikut\\
- a,b: rentang x (default -2,2)\\
- -c,d: rentang y (default: skala dengan nilai)\\
- r: alternatifnya radius di sekitar pusat plot\\
- cx,cy: koordinat pusat plot (default 0,0)

Keterangan:(menggambar grafik fungsi satu variabel yang fungsinya
didefinisikan sebagai fungsi simbolik)\\
- \&: untuk menampilkan variabel pada teks

Berikut adalah beberapa contoh menggunakan fungsi. Seperti biasa di
EMT, fungsi yang berfungsi untuk fungsi atau ekspresi lain, jadi kita
dapat meneruskan parameter tambahan (selain x) yang bukan variabel
global ke fungsi dengan parameter titik koma atau dengan koleksi
panggilan.
\end{eulercomment}
\begin{eulerprompt}
>function f(x,a) := x^2/a+a*x^2-x; // define a function
>a=0.3; plot2d("f",0,1;a): // plot with a=0.3
\end{eulerprompt}
\eulerimg{29}{images/Dida Arkadia Ayu Jawata_22305144005_Plot2D (1)-001.png}
\begin{eulerprompt}
>plot2d("f",0,1;0.4): // plot with a=0.4
\end{eulerprompt}
\eulerimg{29}{images/Dida Arkadia Ayu Jawata_22305144005_Plot2D (1)-002.png}
\begin{eulerprompt}
>plot2d(\{\{"f",0.2\}\},0,1); // plot with a=0.2
>plot2d(\{\{"f(x,b)",b=0.1\}\},0,1): // plot with 0.1
\end{eulerprompt}
\eulerimg{29}{images/Dida Arkadia Ayu Jawata_22305144005_Plot2D (1)-003.png}
\begin{eulerprompt}
>function f(x) := x^3-x;...
>plot2d("f",r=1):
\end{eulerprompt}
\eulerimg{29}{images/Dida Arkadia Ayu Jawata_22305144005_Plot2D (1)-004.png}
\begin{eulercomment}
Berikut merupakan ringkasan dari fungsi yang diterima\\
- ekspresi atau ekspresi simbolik dalam x\\
- fungsi atau fungsi simbolis dengan nama sebagai "f"\\
- fungsi simbolis hanya dengan nama f\\
\end{eulercomment}
\begin{eulerttcomment}
 
\end{eulerttcomment}
\begin{eulercomment}
Fungsi plot2d() juga menerima fungsi simbolis. Untuk fungsi simbolis,
hanya nama saja yang berfungsi.
\end{eulercomment}
\begin{eulerprompt}
>function f(x) &= diff(x^x,x)
\end{eulerprompt}
\begin{euleroutput}
  
                              x
                             x  (log(x) + 1)
  
\end{euleroutput}
\begin{eulerprompt}
>plot2d(f,0,2):
\end{eulerprompt}
\eulerimg{29}{images/Dida Arkadia Ayu Jawata_22305144005_Plot2D (1)-005.png}
\begin{eulerprompt}
>expr &= sin(x)*exp(-x)
\end{eulerprompt}
\begin{euleroutput}
  
                                - x
                               E    sin(x)
  
\end{euleroutput}
\begin{eulerprompt}
>plot2d(expr,0,3pi):
\end{eulerprompt}
\eulerimg{29}{images/Dida Arkadia Ayu Jawata_22305144005_Plot2D (1)-006.png}
\begin{eulerprompt}
>function f(x) &=x^x;
>plot2d(f,r=1,cx=1,cy=1,color=red,thickness=2)
>plot2d(&diff(f(x),x),>add,color=blue,style="-.-"):
\end{eulerprompt}
\eulerimg{29}{images/Dida Arkadia Ayu Jawata_22305144005_Plot2D (1)-007.png}
\begin{eulercomment}
Untuk gaya garis ada berbagai pilihan.\\
- gaya="...". Pilih dari "-","--","-.",".",".-.","-.-".\\
- warna: Lihat di bawah untuk warna.\\
- ketebalan: Default adalah 1.\\
\end{eulercomment}
\begin{eulerttcomment}
 
\end{eulerttcomment}
\begin{eulercomment}
Warna dapat dipilih sebagai salah satu warna default, atau sebagai
warna RGB.\\
- 0.15: indeks warna default.\\
- konstanta warna: putih, hitam, merah, hijau, biru, cyan, zaitun,
abu-abu muda, abu-abu, abu-abu tua, oranye, hijau muda, pirus, biru
muda, oranye terang, kuning\\
- rgb(merah, hijau, biru): parameter adalah real dalam [0,1].
\end{eulercomment}
\begin{eulerprompt}
>plot2d("exp(-x^2)",r=2,color=blue,thickness=3,style="--"):
\end{eulerprompt}
\eulerimg{29}{images/Dida Arkadia Ayu Jawata_22305144005_Plot2D (1)-008.png}
\begin{eulerprompt}
>aspect(2); columnsplot(ones(1,16),lab=0:15,grid=0,color=0:15):
\end{eulerprompt}
\eulerimg{14}{images/Dida Arkadia Ayu Jawata_22305144005_Plot2D (1)-009.png}
\begin{eulerprompt}
>columnsplot(ones(1,16),grid=0,color=rgb(0,0,linspace(0,1,15))):
\end{eulerprompt}
\eulerimg{14}{images/Dida Arkadia Ayu Jawata_22305144005_Plot2D (1)-010.png}
\eulerheading{Menggambar Beberapa Kurva Sekaligus}
\begin{eulercomment}
Dalam subtopik ini, kita akan membahas mengenai cara menggambar
beberapa kurva sekaligus. Dalam hal ini kita dapat menggambar beberapa
kurva dalam jendela grafik yang berbeda secara bersama-sama. Untuk
membuat ini kita dapat menggunakan perintah figure(). Berikut contoh
dari menggambar beberapa kurva sekaligus

Menggambar plot fungsi\\
\end{eulercomment}
\begin{eulerformula}
\[
x^n, 1 \leq n \leq 4
\]
\end{eulerformula}
\begin{eulerprompt}
>reset;
>figure(2,2);...
>for n=1 to 4; figure(n); plot2d("x^"+n); end;...
>figure(0):
\end{eulerprompt}
\eulerimg{29}{images/Dida Arkadia Ayu Jawata_22305144005_Plot2D (1)-011.png}
\begin{eulercomment}
Penjelasan sintaks dari plot fungsi

\end{eulercomment}
\begin{eulerformula}
\[
x^n,  1 \leq n \leq 4
\]
\end{eulerformula}
\begin{eulercomment}
- reset;\\
Perintah ini berguna untuk menghapus grafik yang telah ada sebelumnya,
sehingga kita dapat memulai dari awal untuk menggambar grafik\\
- figure(2x2);\\
Perintah figure() digunakan untuk membuat jendela grafik dengan ukuran\\
axb. Dalam kasus ini perintah figure(2,2) memiliki makna bahwa jendela
grafik yang dibuat berukuran 2x2. Artinya, akan ada empat jendela
grafik yang akan ditampilkan dengan tata letak 2 baris dan 2 kolom.\\
- for n=1 to 4;\\
Perintah ini digunakan untuk melakukan pengulangan (looping) perintah
sebanyak empat kali, yaitu dari 1 hingga 4.\\
- figure(n);\\
Perintah ini digunakan untuk beralih dari jendela grafik satu ke
jendela grafik lainnya (jendela grafik ke-n).\\
- plot2d("x\textasciicircum{}"+n);\\
Perintah plot2d() digunakan untuk membuat plot fungsi matematika.\\
Dalam hal ini fungsi yang diplot adalah x\textasciicircum{}n, di mana n adalah nilai
dari variabel yang sedang diulang. Dengan kata lain, ini akan membuat\\
plot dari x\textasciicircum{}1, x\textasciicircum{}2, x\textasciicircum{}3, dan x\textasciicircum{}4 dalam jendela grafik yang sesuai\\
- end;\\
Perintah ini menandakan akhir dari looping.\\
- figure(0);\\
Perintah ini digunakan untuk beralih kembali ke jendela grafik utama.
\end{eulercomment}
\begin{eulerprompt}
>figure(2,2);... 
>for n=1 to 4; figure(n); plot2d("x^"+n); end;..
\end{eulerprompt}
\begin{eulercomment}
Dari sini dapat kita perhatikan untuk membuat kurva fungsi x\textasciicircum{}n (x
pangkat n) perintahnya tidak ditulis dengan (x\textasciicircum{}n) melainkan ditulis
dengan ("x\textasciicircum{}"+n). Tanda petik dua ("...") digunakan untuk
mengidentifikasi bahwa teks tersebut merupakan ekspresi matematika.\\
Sedangkan tanda (+) digunakan untuk menggabungkan string dengan nilai
yang berubah-ubah atau variabel.

Contoh lain:\\
Menggambar plot fungsi\\
\end{eulercomment}
\begin{eulerformula}
\[
f(x)=x^3-x, -2<x<2
\]
\end{eulerformula}
\begin{eulerprompt}
>reset;
>figure(3,3);...
>for k=1:9; figure(k); plot2d("x^3-x",-2,2,grid=k); end;...
>figure(0):
\end{eulerprompt}
\eulerimg{29}{images/Dida Arkadia Ayu Jawata_22305144005_Plot2D (1)-012.png}
\begin{eulerttcomment}
 Penjelasan sintaks dari plot fungsi
\end{eulerttcomment}
\begin{eulerformula}
\[
f(x)=x^3-x, -2<x<2
\]
\end{eulerformula}
\begin{eulercomment}
- reset;\\
Perintah ini berguna untuk menghapus grafik yang telah ada sebelumnya,
sehingga kita dapat memulai dari awal untuk menggambar grafik\\
- figure (3,3);\\
Perintah ini digunakan untuk membuat jendela grafik dengan ukuran 3x3.
Artinya, akan ada empat jendela grafik yang akan ditampilkan dengan
tata letak 3 baris dan 3 kolom.\\
- for k=1:9;\\
Perintah ini digunakan untuk melakukan pengulangan (looping) perintah
sebanyak sembilan kali.\\
- figure(n);\\
Perintah ini digunakan untuk beralih dari jendela grafik satu ke\\
\end{eulercomment}
\begin{eulerttcomment}
 jendela grafik lainnya (jendela grafik ke-n).
\end{eulerttcomment}
\begin{eulercomment}
- plot2d("x\textasciicircum{}3-x",-2,2,grid=k);\\
Perintah plot2d() digunakan untuk membuat plot fungsi matematika.\\
Dalam hal ini fungsi yang diplot adalah x\textasciicircum{}3-x, dengan batas sumbu x
dari -2 hingga 2. Argumen grid=k digunakan untuk mengaktifkan grid
pada jendela grafik ke-k.\\
- end;\\
Perintah ini menandakan akhir dari looping.\\
- figure(0);\\
Perintah ini digunakan untuk beralih kembali ke jendela grafik utama.

Dari contoh diatas dapat kita perhatikan bahwa tampilan plot dari yang
ke-1 hingga ke-9 memiliki tampilan yang berbeda-beda. Dalam EMT
memiliki berbagai gaya plot 2D yang dapat dijalankan menggunakan
perintah grid=n dimana n adalah jumlah langkah minimal. Setiap nilai n
memiliki tampilan plot adaptif yang berbeda dalam plot 2D, diantaranya
yaitu:\\
0 : tidak ada grid (kisi), frame, sumbu, dan label, hanya kurva saja\\
1 : dengan sumbu, label-label sumbu di luar frame jendela grafik\\
2 : tampilan default\\
3 : dengan grid pada sumbu x dan y, label-label sumbu berada di dalam
jendela grafik\\
4 : tidak ada grid (kisi), sumbu x dan y, dan label berada di luar
frame jendela grafik\\
5 : tampilan default tanpa margin di sekitar plot\\
6 : hanya dengan sumbu x y dan label, tanpa grid\\
7 : hanya dengan sumbu x y dan tanda-tanda pada sumbu.\\
8 : hanya dengan sumbu dan tanda-tanda pada sumbu, dengan tanda-tanda
yang lebih halus pada sumbu.\\
9 : tampilan default dengan tanda-tanda kecil di dalam jendela\\
10: hanya dengan sumbu-sumbu, tanpa tanda

Contoh lain:\\
Menggambar plot fungsi\\
\end{eulercomment}
\begin{eulerformula}
\[
g(x)=2x^3-x
\]
\end{eulerformula}
\begin{eulerprompt}
>reset;
>aspect(1.2);
>figure(3,4); ...
> figure(2); plot2d("2x^3-x",grid=1); ... // x-y-axis
> figure(3); plot2d("2x^3-x",grid=2); ... // default ticks
>figure(4); plot2d("2x^3-x",grid=3); ... // x-y- axis with labels inside
>figure(5); plot2d("2x^3-x",grid=4); ... // no ticks, only labels
>figure(6); plot2d("2x^3-x",grid=5); ... // default, but no margin
>figure(7); plot2d("2x^3-x",grid=6); ... // axes only
>figure(8); plot2d("2x^3-x",grid=7); ... // axes only, ticks at axis
>figure(9); plot2d("2x^3-x",grid=8); ... // axes only, finer ticks at axis
>figure(10); plot2d("2x^3-x",grid=9); ... // default, small ticks inside
>figure(11); plot2d("2x^3-x",grid=10); ...// no ticks, axes only
>figure(0):
\end{eulerprompt}
\eulerimg{24}{images/Dida Arkadia Ayu Jawata_22305144005_Plot2D (1)-013.png}
\begin{eulercomment}
Penjelasan sintaks dari plot fungsi\\
\end{eulercomment}
\begin{eulerformula}
\[
g(x)=2x^3-x
\]
\end{eulerformula}
\begin{eulercomment}
- aspect(1.2);\\
Perintah aspect() digunakan untuk mengatur rasio aspek dari jendela
grafik. Hal ini berarti perintah aspect(1.2); akan menghasilkan plot
dengan perbandingan rasio panjang dan lebar 2:1.\\
- figure(3,4);\\
Perintah ini digunakan untuk membuat jendela grafik dengan ukuran 3x4.\\
Jadi, akan ada total 12 jendela grafik yang akan ditampilkan dalam
tata letak 3 baris dan 4 kolom.\\
- figure(1); plot2d("x\textasciicircum{}3-x",grid=0); ...\\
Adalah perintah untuk beralih ke jendela grafik pertama dan menggambar
plot dari fungsi x\textasciicircum{}3 - x tanpa grid, frame, atau sumbu.\\
- figure(2); plot2d("x\textasciicircum{}3-x",grid=1); ...\\
Adalah perintah untuk beralih ke jendela grafik kedua dan menggambar
plot dari fungsi x\textasciicircum{}3 - x dengan grid hanya pada sumbu x dan y.\\
- figure(3); plot2d("x\textasciicircum{}3-x",grid=2); ...\\
Adalah perintah untuk beralih ke jendela grafik ketiga dan menggambar
plot dari fungsi x\textasciicircum{}3 - x dengan tampilan default, termasuk tanda-tanda
default pada sumbu.\\
- figure(4); plot2d("x\textasciicircum{}3-x",grid=3); ...\\
Adalah perintah untuk beralih ke jendela grafik keempat dan menggambar
plot dari fungsi x\textasciicircum{}3 - x dengan grid pada sumbu x dan y, serta
label-label sumbu yang ada di dalam jendela.\\
- figure(5); plot2d("x\textasciicircum{}3-x",grid=4); ...\\
Adalah perintah untuk beralih ke jendela grafik kelima dan menggambar
plot dari fungsi x\textasciicircum{}3 - x tanpa tanda-tanda sumbu, hanya label-label
yang ada.\\
- figure(6); plot2d("x\textasciicircum{}3-x",grid=5); ...\\
Adalah perintah untuk beralih ke jendela grafik keenam dan menggambar
plot dari fungsi x\textasciicircum{}3 - x dengan tampilan default, tetapi tanpa margin
di sekitar plot.\\
- figure(7); plot2d("x\textasciicircum{}3-x",grid=6); ...\\
Adalah perintah untuk beralih ke jendela grafik ketujuh dan menggambar
plot dari fungsi x\textasciicircum{}3 - x hanya dengan sumbu-sumbu (tanpa grid atau
label).\\
- figure(8); plot2d("x\textasciicircum{}3-x",grid=7); ...\\
Adalah perintah untuk beralih ke jendela grafik kedelapan dan
menggambar plot dari fungsi x\textasciicircum{}3 - x hanya dengan sumbu-sumbu dan
tanda-tanda pada sumbu.\\
- figure(9); plot2d("x\textasciicircum{}3-x",grid=8); ...\\
Adalah perintah untuk beralih ke jendela grafik kesembilan dan
menggambar plot dari fungsi x\textasciicircum{}3 - x hanya dengan sumbu-sumbu dan
tanda-tanda pada sumbu, dengan tanda-tanda yang lebih halus pada
sumbu.\\
- figure(10); plot2d("x\textasciicircum{}3-x",grid=9); ...\\
Adalah perintah untuk beralih ke jendela grafik kesepuluh dan
menggambar plot dari fungsi x\textasciicircum{}3 - x dengan tanda-tanda default kecil
di dalam jendela.\\
- figure(11); plot2d("x\textasciicircum{}3-x",grid=10); ...\\
Adalah perintah untuk beralih ke jendela grafik kesebelas dan
menggambar plot dari fungsi x\textasciicircum{}3 - x hanya dengan sumbu-sumbu, tanpa
tanda-tanda.\\
- figure(0);\\
Adalah perintah untuk beralih kembali ke jendela grafik utama atau
jendela grafik dengan nomor 0 setelah semua perintah dalam urutan
selesai dieksekusi.

Dari ketiga contoh di atas, dapat kita katakan bahwa untuk menggambar
beberapa kurva sekaligus itu dapat dilakukan dengan satu baris
perintah ataupun dengan cara mendefinisikannya 1 per 1.

Terlihat beberapa jenis grid memiliki tampilan yang mirip atau sama,
seperti 1 dan 2, 2 dan 5, 4 dan 9, 7 dan 8, untuk dapat membedakannya
secara lebih jelas, ubah grid dari contoh di bawah ini.
\end{eulercomment}
\begin{eulerprompt}
>reset;
>aspect(1.3);
>figure(1,3);...
>figure (1); plot2d("x^2*exp(-x)",0,10);...
>figure (2); plot2d("2*exp(x)",-5,5);...
>figure (3); plot2d("exp(x^2)",-2,2);...
>figure (0):
\end{eulerprompt}
\eulerimg{22}{images/Dida Arkadia Ayu Jawata_22305144005_Plot2D (1)-014.png}
\begin{eulercomment}
Contoh lain:
\end{eulercomment}
\begin{eulerprompt}
>reset;
>aspect(3/4);
>figure(2,1);...
>for a=1:2; figure(a); plot2d("2*x*log(x^2)",0,3,grid=a); end;...
>figure(0):
\end{eulerprompt}
\eulerimg{34}{images/Dida Arkadia Ayu Jawata_22305144005_Plot2D (1)-015.png}
\begin{eulercomment}
\begin{eulercomment}
\eulerheading{Menggambar Beberapa Kurva}
\begin{eulercomment}
* pada bidang koordinat yang sama

Plot lebih dari satu fungsi (multiple function) ke dalam satu jendela
dapat dilakukan dengan berbagai cara. Salah satu caranya adalah
menggunakan \textgreater{}add untuk beberapa panggilan ke plot2d secara
keseluruhan, kecuali panggilan pertama.

Berikut contohnya:\\
menggambar kurva\\
\end{eulercomment}
\begin{eulerformula}
\[
 f(x)=cos(x)
\]
\end{eulerformula}
\begin{eulerformula}
\[
f(x)= x^2
\]
\end{eulerformula}
\begin{eulerprompt}
>aspect(); plot2d("cos(x)",r=3); plot2d("x^2",style=".",>add):
\end{eulerprompt}
\eulerimg{29}{images/Dida Arkadia Ayu Jawata_22305144005_Plot2D (1)-016.png}
\begin{eulerformula}
\[
f(x)=cos(x)-1
\]
\end{eulerformula}
\begin{eulerformula}
\[
f(x)= sin(x)-1
\]
\end{eulerformula}
\begin{eulerprompt}
>aspect(2); plot2d("cos(x)-1",-1,6); plot2d("sin(x)-1",style="--",>add):
\end{eulerprompt}
\eulerimg{14}{images/Dida Arkadia Ayu Jawata_22305144005_Plot2D (1)-017.png}
\begin{eulercomment}
Selain menggunakan \textgreater{}add kita juga bisa menambahkannya secara langsung

Berikut contohnya:\\
Menggambar kurva\\
\end{eulercomment}
\begin{eulerformula}
\[
f(x)= 2x+1
\]
\end{eulerformula}
\begin{eulerformula}
\[
f(x)= -2x+1
\]
\end{eulerformula}
\begin{eulerprompt}
>plot2d(["2x+1","x"],0,8):
\end{eulerprompt}
\eulerimg{14}{images/Dida Arkadia Ayu Jawata_22305144005_Plot2D (1)-018.png}
\begin{eulerformula}
\[
f(x)=sin(2x)
\]
\end{eulerformula}
\begin{eulerformula}
\[
f(x)=cos(3x)
\]
\end{eulerformula}
\begin{eulerprompt}
>aspect(1.5); plot2d(["sin(2x)","cos(3x)"],0,8):
\end{eulerprompt}
\eulerimg{19}{images/Dida Arkadia Ayu Jawata_22305144005_Plot2D (1)-019.png}
\begin{eulercomment}
Kegunaan \textgreater{}add yang lain juga bisa untuk menambahkan titik pada kurva.

Berikut contohnya:\\
Menambahkan sebuah titik di\\
\end{eulercomment}
\begin{eulerformula}
\[
f(x)= x+4
\]
\end{eulerformula}
\begin{eulerprompt}
>aspect(); plot2d("x+4",-2,5,); plot2d(2,6,>points,>add):
\end{eulerprompt}
\eulerimg{29}{images/Dida Arkadia Ayu Jawata_22305144005_Plot2D (1)-020.png}
\begin{eulercomment}
Kita juga bisa mencari titik perpotongan dengan cara berikut:

\end{eulercomment}
\begin{eulerformula}
\[
sin(x)=2x
\]
\end{eulerformula}
\begin{eulerprompt}
>plot2d(["sin(x)","2x"],r=2,cx=1,cy=1, ...
>  color=[black,blue],style=["-","."], ...
>  grid=1);
>x0=solve("sin(x)-2x",1);  ...
>  plot2d(x0,x0,>points,>add);  ...
>  label("sin(x) = 2x",x0,x0,pos="cl",offset=20):
\end{eulerprompt}
\eulerimg{29}{images/Dida Arkadia Ayu Jawata_22305144005_Plot2D (1)-021.png}
\begin{eulerprompt}
>function f(x,a) := x^2+a*x-x/a; ...
>plot2d("f",-10,10;1,title="a=1"):
\end{eulerprompt}
\eulerimg{29}{images/Dida Arkadia Ayu Jawata_22305144005_Plot2D (1)-022.png}
\begin{eulerprompt}
> plot2d(\{\{"f",1\}\},-10,10); ...
>for a=1:10; plot2d(\{\{"f",a\}\},>add); end:
\end{eulerprompt}
\eulerimg{29}{images/Dida Arkadia Ayu Jawata_22305144005_Plot2D (1)-023.png}
\begin{eulerprompt}
>function f(x,a) := x^2*exp(-x^2/a); ...
>plot2d("f",-10,10;5,thickness=2,title="a=5"):
\end{eulerprompt}
\eulerimg{29}{images/Dida Arkadia Ayu Jawata_22305144005_Plot2D (1)-024.png}
\begin{eulerprompt}
>plot2d(\{\{"f",1\}\},-8,8); ...
>for a=2:5; plot2d(\{\{"f",a\}\},>add,thickness=2); end:
\end{eulerprompt}
\eulerimg{29}{images/Dida Arkadia Ayu Jawata_22305144005_Plot2D (1)-025.png}
\begin{eulerprompt}
>aspect(2.1); &plot2d(1/x,[x,-1,1]):
\end{eulerprompt}
\eulerimg{26}{images/Dida Arkadia Ayu Jawata_22305144005_Plot2D (1)-026.png}
\begin{eulerprompt}
>x=linspace(-1,1,50);...
>plot2d("1/x"):
\end{eulerprompt}
\eulerimg{13}{images/Dida Arkadia Ayu Jawata_22305144005_Plot2D (1)-027.png}
\begin{eulerprompt}
> 
\end{eulerprompt}
\end{eulernotebook}
\end{document}
